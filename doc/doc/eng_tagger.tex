\documentclass[12pt]{article}
\usepackage[latin1]{inputenc}
\usepackage{graphicx}
\usepackage{makeidx}
\usepackage{hyperref}
\usepackage{CJK}
\usepackage{times}
\makeindex
\title{English POS Tagging}
\begin{document}
\begin{CJK}{GBK}{song}
\maketitle

\section{How to compile}
Suppose that Zpar has been downloaded to the directory \textit{zpar}. To make a POS tagging system for English, type \textit{make english.postagger}. This will create a directory \textit{zpar/dist/english.postagger}, in which there are two files: \textit{train} and \textit{tagger}. The file \textit{train} is used to train a tagging model,and the file \textit{tagger} is used to tag new texts using a trained parsing model.
\section{Format of inputs and outputs}
The input files to the \textit{tagger} are formatted as a sequence of tokenized English sentences. An example input is:
\\
\\
\hspace{3cm} Ms. Haag plays Elianti .
\\
\\
The output files contain space-separated words:
\\
\\
\hspace{3cm} Ms./NNP Haag/NNP plays/VBZ Elianti/NNP ./.
\\
\\
The output format is also the format of training files for the \textit{train} file.
\section{How to train a model}
To train a model, use
\\
\\
\hspace{3cm} zpar/dist/english.postagger/train $<$train-file$>$ $<$model-file$>$ $<$number of iterations$>$ \\
\\
For example, using the \href{eng_pos_files/train.txt}{train file}, you can train a  model by
\\
\\
\hspace{3cm} zpar/dist/english.postagger/train train.txt model 1 \\
\\
After training is completed, a new file \textit{model} will be created in the current directory, which can be used to assign POS tags to tokenized sentences. The above command performs training with one iteration using the training file.
\section{How to tag new texts}
To apply an existing model to tag new texts, use
\\
\\
\hspace{3cm} zpar/dist/english.postagger/tagger $<$model$>$ $<$input-file$>$ $<$output-file$>$
\\
\\
For example, using the model we just trained, we can tag \href{eng_pos_files/input.txt}{an example input} by
\\
\\
\hspace{3cm} zpar/dist/english.postagger/tagger model input.txt output.txt
\\
\\
The output file contains automatically tagged sentences.
\section{Outputs and evaluation}
Automatically tagged texts contain errors. In order to evaluate the quality of the outputs, we can manually specify the POS tags of a sample, and compare the outputs with the correct sample.
\\
Manually specified POS tags of the input file are given in \href{eng_pos_files/reference.txt}{reference file}. Here is a \href{eng_pos_files/evaluate.py}{Python script} that performs automatic evaluation.
\\
Using the above \textit{output.txt} and \textit{reference.txt}, we can evaluate the accuracies by typing
\\
\\
\hspace{3cm} python evaluate.py output.txt reference.txt
\\
\\
The output of the evaluation script is a single number: \textit{per-token accuracy} which is defined to be the ratio of correctly POS-tagged words over all the words in output.txt.
\section{How to tune the performance of a system}
The performance of the system after one training iteration may not be optimal. You can try training a model for another few iterations, after each you compare the performance. You can choose the model that gives the highest f-score on your test data. We conventionally call this test file the development test data, because you develop a parsing model using this. Here is a \href{eng_pos_files/test.sh}{a shell script} that automatically trains the POS tagger for 30 iterations, and after the $i$th iteration, stores the model file to model.$i$. You can compare the f-score of all 30 iterations and choose model.$k$, which gives the best f-score, as the final model. In this file, this is a variable called \textit{zpar}. You need to set this variable to the relative directory of \textit{zpar/dist/english.postagger}.
\end{CJK}
\end{document}
